% 
\documentclass[a4paper]{article}
\usepackage{Sweave}
\usepackage{amsmath}
\usepackage{hyperref}
\usepackage[utf8]{inputenc}
\title{optimr: An An Integrated Framework for Optimization}
\author{Emilio L. Cano}

\begin{document}

\maketitle

\tableofcontents

\section{Introduction}

\section{Definitions}
\begin{description}
  \item[entity] An entity is a basic type items in an optimisation problem.
  These main entities are: sets, variables, parameters and equations. Terms are
  a special entity composed by a variable or parameter and further information
  to compound an equation. Constants and Scalars are auxiliary entities for
  modeling.
  \item[item] An item is an specific instance of an entity. It is represented by
  an id and a symbol, and can contain more information as descriptions, indices,
  values or others.
  \item [Set]
  \item [Variable]
  \item [Parameter]
  \item [Equation]
  \item [Term]
  \item [Constant]
  \item [Scalar]
\end{description}

\section{Optimization Workflow}
First, an \texttt{optimSMS} object has to be created. At least a name
(\texttt{name}) must be provided. Other available arguments are short
description (\texttt{sDes}) and long description (\texttt{lDes}):

\begin{Schunk}
\begin{Sinput}
> mySMS <- newSMS("MyFirstSMS")
\end{Sinput}
\end{Schunk}

After its creation, besides the name and descriptions, it contains the data
structures for each entity of the model: sets, variables, parameters, equations
and terms (of equations).

\begin{Schunk}
\begin{Sinput}
> getClass("optimSMS")
\end{Sinput}
\begin{Soutput}
Class "optimSMS" [package "optimr"]

Slots:
                                       
Name:      consts       sets       vars
Class: data.frame data.frame data.frame
                                       
Name:        pars        eqs      terms
Class: data.frame data.frame data.frame
                                       
Name:        name       sDes       lDes
Class:  character  character  character
\end{Soutput}
\begin{Sinput}
> str(mySMS)
\end{Sinput}
\begin{Soutput}
Formal class 'optimSMS' [package "optimr"] with 9 slots
  ..@ consts:'data.frame':	0 obs. of  6 variables:
  .. ..$ id    : logi(0) 
  .. ..$ symbol: logi(0) 
  .. ..$ tag   : logi(0) 
  .. ..$ sDes  : logi(0) 
  .. ..$ lDes  : logi(0) 
  .. ..$ value : logi(0) 
  ..@ sets  :'data.frame':	0 obs. of  8 variables:
  .. ..$ id    : logi(0) 
  .. ..$ symbol: logi(0) 
  .. ..$ tag   : logi(0) 
  .. ..$ sDes  : logi(0) 
  .. ..$ lDes  : logi(0) 
  .. ..$ loc   : logi(0) 
  .. ..$ inSet : logi(0) 
  .. ..$ part  : logi(0) 
  ..@ vars  :'data.frame':	0 obs. of  11 variables:
  .. ..$ id      : logi(0) 
  .. ..$ symbol  : logi(0) 
  .. ..$ tag     : logi(0) 
  .. ..$ sDes    : logi(0) 
  .. ..$ lDes    : logi(0) 
  .. ..$ nature  : logi(0) 
  .. ..$ units   : logi(0) 
  .. ..$ dataType: logi(0) 
  .. ..$ ind     : logi(0) 
  .. ..$ integer : logi(0) 
  .. ..$ positive: logi(0) 
  ..@ pars  :'data.frame':	0 obs. of  10 variables:
  .. ..$ id      : logi(0) 
  .. ..$ symbol  : logi(0) 
  .. ..$ tag     : logi(0) 
  .. ..$ sDes    : logi(0) 
  .. ..$ lDes    : logi(0) 
  .. ..$ nature  : logi(0) 
  .. ..$ units   : logi(0) 
  .. ..$ dataType: logi(0) 
  .. ..$ group   : logi(0) 
  .. ..$ ind     : logi(0) 
  ..@ eqs   :'data.frame':	0 obs. of  8 variables:
  .. ..$ id      : logi(0) 
  .. ..$ symbol  : logi(0) 
  .. ..$ tag     : logi(0) 
  .. ..$ sDes    : logi(0) 
  .. ..$ lDes    : logi(0) 
  .. ..$ nature  : logi(0) 
  .. ..$ relation: logi(0) 
  .. ..$ domain  : logi(0) 
  ..@ terms :'data.frame':	0 obs. of  14 variables:
  .. ..$ id      : logi(0) 
  .. ..$ symbol  : logi(0) 
  .. ..$ tag     : logi(0) 
  .. ..$ sDes    : logi(0) 
  .. ..$ lDes    : logi(0) 
  .. ..$ eq      : logi(0) 
  .. ..$ side    : logi(0) 
  .. ..$ parent  : logi(0) 
  .. ..$ nature  : logi(0) 
  .. ..$ item    : logi(0) 
  .. ..$ setSums : logi(0) 
  .. ..$ power   : logi(0) 
  .. ..$ sign    : logi(0) 
  .. ..$ setSubEq: logi(0) 
  ..@ name  : chr "MyFirstSMS"
  ..@ sDes  : chr ""
  ..@ lDes  : chr ""
\end{Soutput}
\end{Schunk}

Now we can add items to each entity of the model. Before adding an indexed item,
the corresponding set must have been created. Thus, it is better to add first
the sets, then the variables and parameters, and finally the equations and
terms.

Let us add some sets:



\section{Equations}

\subsection{Modified Indices}

A term index can vary dynamically as a function of other indices or parameters.
In this case, the term in the equation must include which (auxiliary) equation
should substitute a given index.

NA

This equations are composed by the set to be substituted on the left side,
and the substituting expression on the right side.

When an index in a term of an equation must be substituted, we add a value for
the variable setSubEq to the term data.frame.

\subsection{Conditional sums}

A sum over a term can be made over a whole set (or defined subset), or over a
conditional statement over sets. Thus, the property \texttt{setSums} of a term
indicates which sets are in the sum, while the property \texttt{condSums} of a
term indicates which conditional equation should be added to the sum. This is
usually used for equations of the type:

\begin{align*}
\sum_{i'<i} x_{i'}\; \forall i \in \mathcal{I}. 
\end{align*}


\section{Problem Instance}

\subsection{Sets}
The sets must be added in the same order than they are in the data.frame of the
sms.

\subsection{Parameter Values}

The list must be in the same order than the indices were saved to the parameter
in the SMS.

\appendix
\section{Optimization Services (OS)}
Implemented the Optimization Services from COIN-OR:

\url{http://www.coin-or.org}

\url{https://projects.coin-or.org/OS}

\url{http://www.optimizationservices.org/}

\end{document}
